\documentclass[12t]{beamer}
\usepackage[utf8]{inputenc}
\usepackage[polish]{babel}
\usepackage{polski}
\usepackage{graphicx}
\usepackage{color}
\input{pygments}

\usetheme{Pittsburgh}

\author{Silesian Ruby Users' Group}
\title{Movie Pole}
\setbeamercovered{transparent}

\begin{document}

\frame{\titlepage}

% nazwiska i adres strony

\begin{frame}[fragile]
  \frametitle{Sposób na nudę}
\begin{verbatim}
10 PRINT "Otwórz http://imdb.com/ lub podobną stronę."
20 PRINT "Szukaj dobrych filmów do obejrzenia."
30 PRINT "Otwórz http://mininova.org/ lub podobną."
40 PRINT "Odszukaj wybrany film."
50 PRINT "Pobierz film używając wybranego programu p2p."
60 PRINT "Obejrzyj film."
70 GOTO 10
\end{verbatim}
\end{frame}

\begin{frame}
  \frametitle{Pomysł}
  Publiczna filmoteka
  \begin{itemize}
  \item wykorzystanie serwisów: IMDb, Mininova, itp.
  \item aplikacja Ruby on Rails - baza filmów, torrentów, kanał RSS
  \item klient desktopowy - czytnik RSS
  \end{itemize}
\end{frame}

\begin{frame}
  \frametitle{IMDb - problemy}
  \begin{itemize}
  \item brak publicznego API
  \item konieczność parsowania HTML
  \item zwracanie danych w przystępnej postaci
  \end{itemize}
\end{frame}

\begin{frame}[fragile]
  \frametitle{Co chcemy uzyskać?}
  \input{imdb}
\end{frame}

\begin{frame}[fragile]
  \frametitle{Co chcemy uzyskać? c.d.}
  \input{fullinformation}
\end{frame}

\begin{frame}[fragile]
  \frametitle{IMDB::NowPlaying}
  \begin{footnotesize}
    \input{nowplaying}
  \end{footnotesize}
\end{frame}

\begin{frame}
  \frametitle{Skupmy się na Ruby on Rails}
  \begin{itemize}
  \item baza danych: filmy, torrenty
  \item dane dodawane z zewnątrz
  \item lista filmów
  \item kanał RSS
  \end{itemize}
\end{frame}

\begin{frame}
  \frametitle{Do dzieła!}
  \begin{itemize}
  \item generujemy aplikację
  \item omówienie struktury
  \item skrypty (serwer, konsola, generatory)
  \item zadania Rake
  \item konfiguracja
  \end{itemize}
\end{frame}

\begin{frame}
  \frametitle{Model Movie}
  \begin{itemize}
  \item generowanie modelu
  \item tworzenie migracji (pola: tytuł, identyfikator IMDb, liczba
    głosów)
  \item dodanie podstawowych walidacji (tytuł, identyfikator IMDb)
  \item oczyszczanie tytułu filmu
  \end{itemize}
\end{frame}

\begin{frame}
  \frametitle{Kontroler Movies}
  \begin{itemize}
  \item generowanie kontrolera
  \item tworzenie akcji index (lista filmów)
  \item tworzenie akcji show (szczegóły filmu)
  \item tworzenie ścieżek (routing)
  \end{itemize}
\end{frame}

\begin{frame}
  \frametitle{Widoki: index, show}
  \begin{itemize}
  \item tworzenie szablonów ERb
  \item nawigacja pomiędzy widokami
  \end{itemize}
\end{frame}

\begin{frame}
  \frametitle{Torrenty}
  \begin{itemize}
  \item tworzenie modelu i migracji
  \item walidacje (rozmiar pliku)
  \item dodanie asocjacji z filmami
  \item uzupełnianie widoków
  \end{itemize}
\end{frame}

\begin{frame}
  \frametitle{Posprzątajmy!}
  \begin{itemize}
  \item podział widoku (partial ``torrent'')
  \item utworzenie metody pomocniczej imdb\_url
  \item utworzenie metody wypisującej rozmiar in\_megabytes
  \end{itemize}
\end{frame}

\begin{frame}
  \frametitle{Dodatek - komentarze}
  \begin{itemize}
  \item generowanie modelu i migracji
  \item walidatory
  \item generowanie kontrolera (akcje new, create)
  \item zagnieżdżanie zasobów
  \item prosty formularz i wyświetlanie
  \item tworzenie layoutu, obsługa flash
  \end{itemize}
\end{frame}

\begin{frame}
  \frametitle{Na zakończenie}
  \begin{itemize}
  \item udostępnienie kanału RSS
  \item integracja aplikacji z backendami
  \end{itemize}
\end{frame}

\begin{frame}
  \frametitle{Aplikacja wxRuby}
  \begin{itemize}
  \item proste GUI
  \item wyświetlanie zawartości kanału RSS
  \item renderowanie HTML-a
  \end{itemize}
\end{frame}

\end{document}
